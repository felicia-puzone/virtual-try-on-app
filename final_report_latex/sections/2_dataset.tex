{\color{gray}\hrule}
\begin{center}
\section{Dataset}
%\textbf{bla bla }
\end{center}
{\color{gray}\hrule}


The dataset used is Dress Code \cite{dress-code}, introduced recently by the AImageLab of UNIMORE. Its main characteristics are:
\begin{itemize}[noitemsep]

\item pubicly available;

\item high-resolution images ($1024 \times 768$);

\item very large dataset with respect of the publicly available ones, with about $50k$ image pairs of try-on garments and corresponding catalog images where each item is worn by a model;

\item multi-category clothes: front-view and full-body of upper-body, lower-body, and full-body;

\item rich annotations: for each cloth there are the dense-pose map, cloth worn by a model, keypoints of the model, label map representing the body-parts segmentation and the body skeleton.

\end{itemize}

