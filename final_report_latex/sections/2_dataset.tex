{\color{gray}\hrule}
\begin{center}
\section{Dataset and Proposed Architectures}
%\textbf{bla bla }
\end{center}
{\color{gray}\hrule}

The dataset used is Dress Code \cite{dress-code}, recently introduced by the AImageLab of UNIMORE. It possesses several key characteristics:

\begin{itemize}[noitemsep]

\item Publicly available.

\item High-resolution images ($1024 \times 768$).

\item Very large dataset compared to publicly available ones, with approximately $50,000$ image pairs of try-on garments and corresponding catalog images, where each item is worn by a model.

\item Multi-category clothes: front-view and full-body of upper-body, lower-body, and full-body attire.

\item Rich annotations: For each piece of clothing, there are dense-pose maps, clothing worn by a model, keypoints of the model, label maps representing body-part segmentation, and the body skeleton.

\end{itemize}

\subsection{Dataset Preprocessing}
% Sistemare guarda word doc resume_of_project_work_CV
 fwkmasldkmsòdlfkmsdòfknsdfgkjnsdfgkljsdfngklsjfgnsdkfjnslkjn
 
 
\begin{algorithm}
\caption{Your Pseudocode Caption}
\begin{algorithmic}[1] % This number sets the line numbering style
\Procedure{YourProcedureName}{}
    \State Initialize variables
    \While{condition}
        \State Do something
        \If{some condition}
            \State Do something else
        \Else
            \State Do another thing
        \EndIf
    \EndWhile
    \State \Return Result
\EndProcedure
\end{algorithmic}
\end{algorithm}
